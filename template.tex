%    \begin{macrocode}
	\newcommand{\pl}[2]{\ifnumequal{#2}{0}{}{\ifnumequal{#2}{1}{#1}{#1^{#2}}}}
%    \end{macrocode}
%\begin{macro}{\PM}
%\changes{v1.0}{2022/08/1}{Versione iniziale parte letterale}
%    \begin{macrocode}
\newcommand{\PM}[5][]{
\pgfkeys{/mymonomio, default, #1}%
\pgfkeys{/myerrori, default}%
\ifnumequal{#2}{0}{}{\ifnumequal{#2}{1}{+\pl{\mypmA}{#3}\pl{\mypmB}{#4}\pl{\mypmC}{#5}}
{\ifnumequal{#2}{-1}{-\pl{\mypmA}{#3}\pl{\mypmB}{#4}\pl{\mypmC}{#5}}
{\ifnumgreater{#2}{0}{+#2\pl{\mypmA}{#3}\pl{\mypmB}{#4}\pl{\mypmC}{#5}}{#2\pl{\mypmA}{#3}\pl{\mypmB}{#4}
\pl{\mypmC}{#5}}}}}}
%    \end{macrocode}
%\end{macro}
%
%\DescribeMacro{\PM}
%
%Comando per scrivere un monomio nella forma
%
% $ka^nb^mc^q$
%
%Se i coefficienti  sono zero non viene stampato nulla
%
%\cs{PM}
%\oarg	{pma = a, pmb = b, pmc =c}
%\marg{k}
%\marg{n}
%\marg{m}
%\marg{q}
%
%Esempi
%
% \iffalse
%<*verb>
% \fi
\begin{verbatim}
	1)$\PM{1}{0}{3}{4}$	
	
	2)$\PM{-2}{2}{0}{4}$
	
	3)$\PM[pma = x, pmb = y, pmc =z]{-2}{-2}{3}{4}$
\end{verbatim}
%\iffalse
%</verb>
% \fi
%
%1)$\PM{1}{0}{3}{4}$	
%
%2)$\PM{-2}{2}{0}{4}$
%
%3)$\PM[pma = x, pmb = y, pmc =z]{-2}{-2}{3}{4}$
%